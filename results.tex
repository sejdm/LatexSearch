\documentclass{article}
\usepackage{amsmath}
\usepackage{amsthm}
\usepackage{amssymb}

\newtheorem{lemma}{Lemma}
\newtheorem{theorem}{Theorem}
\newtheorem{corollary}{Corollary}
\newtheorem{definition}{Definition}
\newtheorem{example}{Example}
\newtheorem{question}{Question}
\theoremstyle{remark}
\newtheorem{remark}{Remark}
\begin{document}

Let $\{a_1\, ,\cdots\, , a_g\, , b_1\, , \cdots\, b_g\}$ be simple curves in $X$
representing the generators of 
$\mathrm{H}_1(X,\, \mathbb{Z})$. Consider a basis $\omega_i\,\in\, \mathrm{H}^0(X,\,
\Omega)$, where $\Omega$ is the holomorphic cotangent bundle of $X$, such that 
$\int_{a_i} \omega_j\,=\,\delta_{ij}$, $i\, ,j\,=\,1\, ,\cdots\, ,g$. The $2g$ vectors
$(\int_{a_i}\omega_1\, , \cdots\, , \int_{a_i} \omega_g)$ and $(\int_{b_j} \omega_1\, , 
\cdots\, , \int_{b_j} \omega_g)$, $1\,\leq\, i\, , j\,\leq\, g$, of $\mathbb{C}^g$ are 
linearly independent over $\mathbb{R}$ and generate a lattice which will be denoted by 
$\Lambda$. Denote by $\mathrm{Jac}(X)$ the complex torus $\mathbb{C}^g/\Lambda$ of
dimension $g$. 
For a fixed point $p_0$, define the map
$$\mu\, :\, X\,\longrightarrow\, \mathrm{Jac}(X)\, ,\,\  ~ 
p\,\longmapsto\, (\int_{p_0}^p \omega_1\, , \cdots\, , \int_{p_0}^p 
\omega_g)\, .$$ This map can be extended by linearity to the space of all divisors. When it 
is restricted to the space of degree~0 divisors, it is a bijection up to linear 
equivalence, implying that is isomorphic to $\mathrm{Pic}^0(X)$~\cite{arbarello1985geometry}. The Jacobian of $X$ is 
isomorphic to $\mathrm{Jac}(X)\,:=\,\mathbb{C}^g/\Lambda$. A different choice of $a_i$ and 
$b_j$ will result in an isomorphic $g$ dimensional complex torus.

\begin{proof}
  Fix a real point $p\,\in\, X$. Consider a map $$\theta\,:\, \mathrm{Sym}^g(X) \,\longrightarrow\, \mathrm{Pic}^0(X)
\, ,\,\ (x_1\, ,\cdots\, ,x_g) \,\longmapsto\, [\displaystyle\sum_{i=1}^g (x_i-p)]\, .$$ Given an element $[D]$
of  $\mathrm{Pic}^0(D)$, its inverse under $\theta$ is the set of effective divisors linearly equivalent to the 
divisor $D+g.p$. By the Riemann--Roch theorem, this linear system, which we will denote by $\mathcal{L}(D+g.p)$, is always
non-empty, and therefore $\theta$ is surjective.

Now consider an element $[D]\,\in\, \mathbb{R}\mathrm{Pic}^0(X)$. Since the real 
part of the curve is non-empty, the line bundle $[D]$ cannot have a quaternionic 
structure since there are no real quaternionic bundles of odd rank. Therefore, the 
line bundle has a real structure, i.e., there is a lift $\widetilde{\sigma}$ of 
$\sigma$ such that $\widetilde{\sigma} \circ \widetilde{\sigma}$ is the identity. 
For any holomorphic section $s$ in $\mathcal{L}(D+g.p)$, the section 
$s+\widetilde{\sigma}(s)$ is fixed by $\widetilde{\sigma}$. The divisor $(s)$, 
denoting the divisor of the zeros of this holomorphic section $s$ is therefore in 
$\mathbb{R}\mathrm{Sym}^{g}(X)$. Therefore, $\theta$ restricted to 
$\mathbb{R}\mathrm{Sym}^{g}(X)$, surjects onto $\mathbb{R}\mathrm{Pic}^0(X)$. Hence 
$\mathrm{Sym}^g(X)$ and $\mathrm{Pic}^0(X)$ have the same number of components.
\end{proof}

\begin{proof}
  Choose a $d\,\geq\, g$ so that $d$ is even. Since $d$ is even, there exists a point
$(z_1\, ,z_2\, ,\cdots\, ,z_d)$
in $\mathrm{Sym}^d(X)$ which is fixed by $\sigma$ and $z_{2k}\,=\,z_{2k-1}$.  
   Consider the map 
   \[\theta\,:\, \mathrm{Sym}^d(X) \,\longrightarrow\, \mathrm{Pic}^0(X)
   \, ,\,\ (x_1\, ,\cdots\, ,x_d) \,\longmapsto\, [\displaystyle\sum_{i=1}^d (x_i-z_i)]\, .\]
Once again, given an element $[D]$ of  $\mathrm{Pic}^0(D)$, its inverse under $\theta$ is the set of effective divisors linearly equivalent to the divisor $D+\sum z_i$. Again, by the Riemann--Roch theorem, the linear system $\mathcal{L}(D+\sum z_i)$, is always non-empty, and therefore $\theta$ is surjective.

Now consider two elements $[D_1]\, ,[D_2]\,\in\, \mathbb{R}\mathrm{Pic}^0(X)$ satisfying
the condition that both of of them have a real structure, i.e., there is a lift
$\widetilde{\sigma}$ of 
$\sigma$ such that $\widetilde{\sigma} \circ\widetilde{\sigma}$ is the identity map. As 
before, a holomorphic section of the form $s_i+\widetilde{\sigma}(s_i)$ in $[D_i]$ 
($i=1,2$) is fixed by $\widetilde{\sigma}$ and therefore, the divisors $(s_i)$ 
associated to this holomorphic section are in $\mathbb{R}\mathrm{Sym}^{g}(X)$. The 
divisor $(s_1)$ is of the form $(z_1,\sigma(z_1),z_2,\sigma(z_2),\ldots 
,z_{d/2},\sigma(z_{d/2})$ for some $z_i$ that are not fixed by $\sigma$. Similarly, 
the divisor $(s_2)$ is of the form $(w_1,\sigma(w_1),w_2,\sigma(w_2),\cdots 
,w_{d/2},\sigma(w_{d/2})$ for some $w_i$ that are not fixed by $\sigma$. Paths 
joining $z_i$ with $w_i$ will induce a path joining $(s_1)$ with $(s_2)$. Therefore, 
any two degree~0 line bundles with a real structure can be connected by a path.

Note that the tensor product $L_1 \otimes L_2$ of two quaternionic line bundles 
$L_1$ and $L_2$ is a real line bundle. Therefore, if there is a quaternionic line 
bundle $L \in \mathrm{Pic}^0(X)$, then the map $L'\,\longrightarrow\, L'\otimes L$ defines a 
bijection between real line bundles and quaternionic line bundles in 
$\mathrm{Pic}^0(X)$.  If there exist line bundles $[D_1],[D_2]\,\in\, 
\mathbb{R}\mathrm{Pic}^0(X)$ with quaternionic structures, then by the previous 
paragraph, there is a path joining $[D_1]\otimes L$ and $[D_2]\otimes L$ which can 
be pulled back via the map $L'\,\longrightarrow\, L'\otimes L$ to a path joining $[D_1]$ and 
$[D_2]$.
\end{proof}

\begin{remark}
  Effective divisors will be seen later to correspond to line bundles with a holomorphic section.
\end{remark}

Let $\{a_1\, ,\cdots\, , a_g\, , b_1\, , \cdots\, b_g\}$ be simple curves in $X$
representing the generators of 
$\mathrm{H}_1(X,\, \mathbb{Z})$. Consider a basis $\omega_i\,\in\, \mathrm{H}^0(X,\,
\Omega)$, where $\Omega$ is the holomorphic cotangent bundle of $X$, such that 
$\int_{a_i} \omega_j\,=\,\delta_{ij}$, $i\, ,j\,=\,1\, ,\cdots\, ,g$. The $2g$ vectors
$(\int_{a_i}\omega_1\, , \cdots\, , \int_{a_i} \omega_g)$ and $(\int_{b_j} \omega_1\, , 
\cdots\, , \int_{b_j} \omega_g)$, $1\,\leq\, i\, , j\,\leq\, g$, of $\mathbb{C}^g$ are 
linearly independent over $\mathbb{R}$ and generate a lattice which will be denoted by 
$\Lambda$. Denote by $\mathrm{Jac}(X)$ the complex torus $\mathbb{C}^g/\Lambda$ of
dimension $g$. 
For a fixed point $p_0$, define the map
$$\mu\, :\, X\,\longrightarrow\, \mathrm{Jac}(X)\, ,\,\  ~ 
p\,\longmapsto\, (\int_{p_0}^p \omega_1\, , \cdots\, , \int_{p_0}^p 
\omega_g)\, .$$ This map can be extended by linearity to the space of all divisors. When it 
is restricted to the space of degree~0 divisors, it is a bijection up to linear 
equivalence, implying that is isomorphic to $\mathrm{Pic}^0(X)$~\cite{arbarello1985geometry}. The Jacobian of $X$ is 
isomorphic to $\mathrm{Jac}(X)\,:=\,\mathbb{C}^g/\Lambda$. A different choice of $a_i$ and 
$b_j$ will result in an isomorphic $g$ dimensional complex torus.

\begin{proof}
  Fix a real point $p\,\in\, X$. Consider a map $$\theta\,:\, \mathrm{Sym}^g(X) \,\longrightarrow\, \mathrm{Pic}^0(X)
\, ,\,\ (x_1\, ,\cdots\, ,x_g) \,\longmapsto\, [\displaystyle\sum_{i=1}^g (x_i-p)]\, .$$ Given an element $[D]$
of  $\mathrm{Pic}^0(D)$, its inverse under $\theta$ is the set of effective divisors linearly equivalent to the 
divisor $D+g.p$. By the Riemann--Roch theorem, this linear system, which we will denote by $\mathcal{L}(D+g.p)$, is always
non-empty, and therefore $\theta$ is surjective.

Now consider an element $[D]\,\in\, \mathbb{R}\mathrm{Pic}^0(X)$. Since the real 
part of the curve is non-empty, the line bundle $[D]$ cannot have a quaternionic 
structure since there are no real quaternionic bundles of odd rank. Therefore, the 
line bundle has a real structure, i.e., there is a lift $\widetilde{\sigma}$ of 
$\sigma$ such that $\widetilde{\sigma} \circ \widetilde{\sigma}$ is the identity. 
For any holomorphic section $s$ in $\mathcal{L}(D+g.p)$, the section 
$s+\widetilde{\sigma}(s)$ is fixed by $\widetilde{\sigma}$. The divisor $(s)$, 
denoting the divisor of the zeros of this holomorphic section $s$ is therefore in 
$\mathbb{R}\mathrm{Sym}^{g}(X)$. Therefore, $\theta$ restricted to 
$\mathbb{R}\mathrm{Sym}^{g}(X)$, surjects onto $\mathbb{R}\mathrm{Pic}^0(X)$. Hence 
$\mathrm{Sym}^g(X)$ and $\mathrm{Pic}^0(X)$ have the same number of components.
\end{proof}

\begin{proof}
  Choose a $d\,\geq\, g$ so that $d$ is even. Since $d$ is even, there exists a point
$(z_1\, ,z_2\, ,\cdots\, ,z_d)$
in $\mathrm{Sym}^d(X)$ which is fixed by $\sigma$ and $z_{2k}\,=\,z_{2k-1}$.  
   Consider the map 
   \[\theta\,:\, \mathrm{Sym}^d(X) \,\longrightarrow\, \mathrm{Pic}^0(X)
   \, ,\,\ (x_1\, ,\cdots\, ,x_d) \,\longmapsto\, [\displaystyle\sum_{i=1}^d (x_i-z_i)]\, .\]
Once again, given an element $[D]$ of  $\mathrm{Pic}^0(D)$, its inverse under $\theta$ is the set of effective divisors linearly equivalent to the divisor $D+\sum z_i$. Again, by the Riemann--Roch theorem, the linear system $\mathcal{L}(D+\sum z_i)$, is always non-empty, and therefore $\theta$ is surjective.

Now consider two elements $[D_1]\, ,[D_2]\,\in\, \mathbb{R}\mathrm{Pic}^0(X)$ satisfying
the condition that both of of them have a real structure, i.e., there is a lift
$\widetilde{\sigma}$ of 
$\sigma$ such that $\widetilde{\sigma} \circ\widetilde{\sigma}$ is the identity map. As 
before, a holomorphic section of the form $s_i+\widetilde{\sigma}(s_i)$ in $[D_i]$ 
($i=1,2$) is fixed by $\widetilde{\sigma}$ and therefore, the divisors $(s_i)$ 
associated to this holomorphic section are in $\mathbb{R}\mathrm{Sym}^{g}(X)$. The 
divisor $(s_1)$ is of the form $(z_1,\sigma(z_1),z_2,\sigma(z_2),\ldots 
,z_{d/2},\sigma(z_{d/2})$ for some $z_i$ that are not fixed by $\sigma$. Similarly, 
the divisor $(s_2)$ is of the form $(w_1,\sigma(w_1),w_2,\sigma(w_2),\cdots 
,w_{d/2},\sigma(w_{d/2})$ for some $w_i$ that are not fixed by $\sigma$. Paths 
joining $z_i$ with $w_i$ will induce a path joining $(s_1)$ with $(s_2)$. Therefore, 
any two degree~0 line bundles with a real structure can be connected by a path.

Note that the tensor product $L_1 \otimes L_2$ of two quaternionic line bundles 
$L_1$ and $L_2$ is a real line bundle. Therefore, if there is a quaternionic line 
bundle $L \in \mathrm{Pic}^0(X)$, then the map $L'\,\longrightarrow\, L'\otimes L$ defines a 
bijection between real line bundles and quaternionic line bundles in 
$\mathrm{Pic}^0(X)$.  If there exist line bundles $[D_1],[D_2]\,\in\, 
\mathbb{R}\mathrm{Pic}^0(X)$ with quaternionic structures, then by the previous 
paragraph, there is a path joining $[D_1]\otimes L$ and $[D_2]\otimes L$ which can 
be pulled back via the map $L'\,\longrightarrow\, L'\otimes L$ to a path joining $[D_1]$ and 
$[D_2]$.
\end{proof}

\begin{proof}
  The map $u:\mathrm{Sym}^n(X) \to \mathrm{Pic}^0(X)$ ($n\geq 2g-2)$ defined by $u(x_1+x_2+\ldots+x_n) = \mathcal{O}(\sum (x_i - p))$, where $p$ is a real point of the curve, is a complex projective bundle, where the fibre of $\mathcal{O}(D)$ is the projectivised linear system $\mathbb{P}\mathcal{L}(D+n.p)$. 

  We now prove that the restriction of $u$ to $\mathrm{Sym}^n(X)^{\sigma}$ surjects onto $\mathrm{Pic}^0(X)^{\sigma}$. Consider an element $\mathcal{O}(D)\,\in\, \mathrm{Pic}^0(X)^{\sigma}$. Since the real part of the curve is non-empty, the line bundle $\mathcal{O}(D)$ cannot have a quaternionic structure since there are no real quaternionic bundles of odd rank. Therefore, the line bundle has a real structure, i.e., there is a lift $\widetilde{\sigma}$ of $\sigma$ such that $\widetilde{\sigma} \circ \widetilde{\sigma}$ is the identity.  For any holomorphic section $s$ in $\mathcal{L}(D+n.p)$, the section $s+\widetilde{\sigma}(s)$ is fixed by $\widetilde{\sigma}$. The divisor $(s)$, denoting the divisor of the zeros of this holomorphic section $s$ is therefore in $\mathrm{Sym}^{n}(X)^{\sigma}$. Therefore, $u$ restricted to $\mathrm{Sym}^n(X)^{\sigma}$, surjects onto $\mathrm{Pic}^0(X)^{\sigma}$. 
  
  The pull-back of a principal divisor  $(f)$ via the complex conjugation involution $\sigma$ is the principal divisor $(\overline{\sigma^*(f)})$. Therefore if $D$ is real, $\sigma^*$ preserves the linear system $\mathcal{L}(D+n.p)$ and, therefore preserves the fibre of a $u(D)$ for a real $D$. Since $u$ is surjective, as proved in the previous paragraph, the fibre of of the restriction of $u$ to the real part is preserved by the involution. 

  Therefore, on restricting to the real part of $\mathrm{Sym}^n(X)$, $u$ is a real projective bundle over the real part of $\mathrm{Pic}^0(X)$. 

By~\cite{biswas}, if $X$ is an M-variety then $\mathrm{Pic}^0(X)$ is also an M-variety. The complex projective space is also an M-variety. Therefore, by the Leray-Hirsch theorem, $\mathrm{Sym}^n(X)$ is also an M-variety.
\end{proof}

\end{document} 
